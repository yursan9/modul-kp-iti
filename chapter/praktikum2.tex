\documentclass[../main.tex]{subfiles}

\begin{document}
\chapter{Pengaturan Alur Program - IF}
\section{Motivasi}
Secara umum, program biasanya mengeksekusi perintah dari awal dan akhir secara
berurutan. Pernyataan pertama dieksekusi, lalu yang kedua, terus yang ketiga,
sampai program mencapai akhirnya. Sebuah program mungkin tidak akan berguna,
jika dia hanya bisa menjalankan perintah yang sama setiap kali dijalankan.
Program akan lebih berguna lagi jika dia bisa mengeksekusi perintah lain,
tergantung dari kriteria yang ditentukan.

Sebagai contoh, jika program untuk memeriksa sebuah dokumen dan mencari berapa
banyak sebuah kata muncul di dalam dokumen tersebut, akan lebih bagus meskipun
dokumennya berubah program tersebut tetap menghasilkan perhitungan yang benar.
Contoh lainnya, sebuah permainan video yang bisa menggerakkan karakter yang ada
sesuai dengan keinginan pemain. Kita perlu program yang bisa diatur alur
eksekusinya. Karena itu kita akan belajar tentang pengaturan alur eksekusi program.

\section{Kondisi}
Pengaturan alur eksekusi program biasanya terjadi karena ada kondisi yang
terpenuhi dalam program tersebut. Biasanya kita akan memeriksa sebuah nilai
atau variabel, apakah nilai atau variabel tersebut memenuhi kondisi yang kita
inginkan?

Untuk memeriksa kondisi tersebut kita bisa memeriksa nilai dengan operator
relasi atau logika.

\subsection{Operator Relasi}
Operator relasi bekerja dengan membandingkan dua ekspresi:
\begin{table}[h!]
\centering
\begin{tabular}{@{} c l @{}}
  \toprule
  Operator  & Arti  \\
  \midrule
  >       	& Lebih dari\\
  >=     		& Lebih dari sama dengan\\
  <      		& Kurang dari\\
  <=    		& Kurang dari sama dengan\\
  ==      	& Sama dengan\\
  !=				&	Tidak sama dengan\\
  \bottomrule
\end{tabular}
\caption{Daftar operator relasi}
\label{ope-rel}
\end{table}

Operator ini sama seperti operator aritmatika (contoh, \code{a > b}) tetapi
mereka menghasilkan nilai \eng{Boolean}; \code{true} atau
\code{false}. Sebagai contoh, anggap variabel \code{x} dan \code{y}
memiliki nilai 4 dan 2. Lalu, \code{x > y} akan menghasilkan \code{true}
dan \code{x < y} akan menghasilkan \code{false}.

\subsection{Operator Logika}
Operator logika biasanya digunakan untuk menggabungkan hasil operasi dari
operator relasi untuk menghasilkan kondisi yang rumit:
\begin{table}[h!]
\centering
\begin{tabular}{@{} c l @{}}
  \toprule
  Operator  & Arti  \\
  \midrule
  \&\&       	& Dan (\eng{and})\\
  ||     		& Atau (\eng{or})\\
  !     		& Bukan (\eng{not})\\
  \bottomrule
\end{tabular}
\caption{Daftar operator logika}
\label{ope-logika}
\end{table}

Operator diatas menghasilkan \code{true} atau \code{false}, menurut
aturan logika. Contoh penggunaan operator logika (jika \code{x=6} dan
\code{y=8}):
\begin{minted}{cpp}
!(x > 2)                        // false
(x == y) || (x < y)             // true
(x == y) && (x > y)             // false
!(x > 2) && (x == y) || (x < y) // true
\end{minted}

\section{\eng{if}, \eng{if-else}, dan \eng{if-else if}}
Bentuk utama dari pengaturan alur \eng{if} adalah:
\begin{minted}{cpp}
if (kondisi){
    pernyataan 1;
    pernyataan 2;
    ...
}
\end{minted}

Jika pernyataannya hanya satu kita bisa menghilangkan tanda kurung kurawal dalam
if, sehingga menghasilkan bentuk:
\begin{minted}{cpp}
if (kondisi)
    pernyataan 1;
\end{minted}

Perhatikan program sederhana untuk menentukan bilangan genap berikut:

\cppfile{code/genap.cpp}

Saat kalian jalankan, lihat hasil program tersebut\ldots{}bilangan yang
dimasukkan mempengaruhi lajurnya program. Jika nilai \code{x} di modulo \(2\)
tidak menghasilkan \(0\), baris ke-12 tidak akan pernah di eksekusi.

Bagaimana kalau program sebelumnya ingin dimodifikasi sehingga program tersebut
juga bisa menentukan bilangan yang dimasukkan itu bilangan bulat atau ganjil?
Kita hanya perlu menambahkan pernyataan \code{else} dalam program.

Sehingga program menjadi seperti ini:

\cppfile{code/genap_ganjil.cpp}

Dalam rangkaian \code{if}, dan nanti \code{else if}, \code{else} berguna
untuk menangkap alur program jika kondisi dalam \code{if} dan \code{else if}
tidak terpenuhi.

Program yang biasa kita gunakan sehari-hari, biasanya memiliki banyak variasi
eksekusi yang dapat dilakukan jika kita menginginkannya. Dengan pernyataan
\eng{if} di atas, kita bisa merubah jalur eksekusi jika kondisinya hanya
terdiri dari satu kondisi. Sekarang kita akan melihat program yang dapat
menerima rantai kondisi yang berbeda. Dibawah ini akan ada program yang memeriksa
umur orang yang datang ke sebuah tempat;

\cppfile{code/penjaga_pintu.cpp}

Jika kita lihat kode program diatas, ada sebuah pernyataan \eng{else if} yang
berguna untuk memberikan kondisi baru sesudah pernyataan \eng{if} sebelumnya.
Ini menyebabkan pernyataan \eng{if, else if,} dan \eng{else} menjadi sebuah
rangkaian kondisi. Rangkaian kondisi harus selalu dimulai dengan pernyataan
\eng{if}. Setelahnya, baru dapat menggunakan \eng{else if} atau diakhiri
oleh \eng{else}.

Program diatas akan memeriksa umur dari pengguna. Jika \(\code{umur} \geq 18\)
maka program mempersilahkan pengguna masuk, jika \(13 < \code{umur} < 18\)
maka pengguna hanya boleh masuk dengan ada bimbingan orang tua, dan jika semua
kondisi sebelumnya tidak ada yang terpenuhi, pengguna dilarang masuk.

\paragraph{Latihan}
\begin{enumerate}
  \item Buatlah sebuah program yang menerima masukkan tahun masehi, dan menentukan
  apakah tahun tersebut tahun kabisat atau bukan.
  \item Buatlah sebuah program yang menghitung nilai akhir dengan ketentuan
  sebagai berikut:
  \begin{verse}
    A : 80 -- 100\\
    B : 70 -- 79\\
    C : 60 -- 69\\
    D : 45 -- 59\\
    E : 0  -- 44\\
  \end{verse}
  Nilai yang dimasukkan berupa nilai UAS, UTS, tugas, dan absensi. Rumus untuk
  nilai keseluruhan:
  \[
    N = 10\% \times \textrm{absensi} + 15\% \times \textrm{tugas} + 25\% \times
    \textrm{UTS} + 50\% \times \textrm{UAS}
  \]
  Tampilkan \(N\) dalam bentuk huruf seperti ketentuan diatas.
\end{enumerate}
\end{document}

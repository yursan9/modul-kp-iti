\documentclass[../main.tex]{subfiles}

\begin{document}
\chapter{Fungsi}
\section{Motivasi}
\eng{Programmer} adalah orang yang paling tidak suka menulis ulang solusi dari
suatu permasalahan. Mereka lebih memilih untuk membuat solusi yang bisa
digunakan secara berulang (bahkan diotomatisasikan). Cara berpikir ini yang
memotivasi penggunaan fungsi dalam bahasa pemrograman.

Jika kita melihat kode yang kita tulis di modul-modul sebelumnya, ada beberapa
yang logikanya ditulis ulang. Seharusnya kita dapat menuliskannya sekali,
dan memakainya berulang kali.

\section{Anatomi Fungsi}

Fungsi biasanya terdiri dari nama fungsi, tipe fungsi, parameter fungsi, dan
nilai kembali. Penulisannya mungkin berbeda disetiap bahasa pemrograman, namun
intinya fungsi itu terdiri dari hal yang disebutkan di atas.

\begin{minted}{cpp}
float kuadrat(float n){
	float hasil = n * n;
	return hasil
}
\end{minted}

Potongan kode di atas menunjukkan cara membuat fungsi sederhana untuk
mengkuadratkan bilangan \(n\). Jika dibedah penulisannya akan menghasilkan;

\begin{description}
	\item[nama fungsi] adalah kuadrat.
	\item[tipe fungsi] adalah \code{int}. Tipe fungsi mempengaruhi tipe data
	yang bisa dijadikan nilai kembali oleh fungsi tersebut.
	\item[parameter fungsi] adalah variabel \code{n} yang bertipe \eng{integer}.
	Parameter fungsi ini bersifat opsional.
	\item[nilai kembali] adalah variabel \code{hasil}. Nilai kembali harus
	sejenis dengan tipe fungsi seperti yang sudah diberitahu di atas.
\end{description}

Fungsi juga bisa digunakan untuk melakukan sesuatu tanpa nilai kembali. Gunakan
tipe data \code{void} sebagai tipe fungsi untuk fungsi yang tidak menghasilkan
nilai kembali.

\begin{minted}{cpp}
void hello(){
	cout << "Hello World!";
}
\end{minted}

\section{Deklarasi dan Inisialisasi}
Terdapat dua bentuk deklarasi dan inisialisasi fungsi, dicontohkan oleh program
dibawah;

\cppfile{code/fungsi_tukar.cpp}

Program di atas melakukan deklarasi dan inisialisasi fungsi secara bersamaan.

\cppfile{code/fungsi_sum.cpp}

Program di atas melakukan deklarasi dan inisialisasi secara terpisah. Di mana
fungsi dikenalkan dulu sebelum fungsi \code{main()}. Untuk parameter saat
perkenalaan tidak perlu disebutkan namanya juga tidak apa, yang terpenting tipe
variabel parameternya sudah diberikan.

\paragraph{Latihan}
\begin{enumerate}
	\item Buat sebuah fungsi yang digunakan untuk mencari rata-rata, nilai
	maksimum dan nilai minimum dari kumpulan nilai \eng{array integer}. Berikan
	contoh penggunaannya pada fungsi \code{main()}.
\end{enumerate}

\end{document}

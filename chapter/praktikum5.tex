\documentclass[../main.tex]{subfiles}

\begin{document}
\chapter{Pengulangan Proses - While}
\section{\eng{While}}
Pada struktur pengulangan \eng{while}, aksi akan terus dijalankan berulang kali sepanjang kondisi \eng{boolean} masih bernilai \eng{True}. Jika kondisi \eng{boolean} bernilai \eng{False}, aksi tidak akan dilaksanakan.

Cara penulisan struktur \eng{while} adalah :

\begin{minted}{cpp}
while (kondisi){
	pernyataan;
}
\end{minted}

Program dibawah akan menampilkan bilangan acak 1-100. Jika \code{i} kurang dari 10, program akan menampilkan bilangan acak sampai \code{i} tidak lagi kurang dari 10.

\cppfile{code/random_10.cpp}

\section{\eng{Do-While}}
Pada struktur pengulangan \eng{do-while}, pernyataan akan terus dijalankan berulang kali sepanjang kondisi \eng{boolean} masih bernilai \eng{True}. Jika kondisi \eng{boolean} bernilai \eng{False}, pernyataan tidak akan dilaksanakan. Yang membedakan antara perintah pengulangan \eng{while} dan \eng{do-while} adalah, pada struktur pengulangan \eng{do-while} perintah akan dilakukan minimal 1 kali baik \eng{boolean} bersifat \eng{True} atau \eng{False}, sedangkan pada \eng{while} jika \eng{boolean} bernilai \eng{False}, maka aksi tidak akan dilakukan sama sekali.

Cara penulisan struktur \eng{do-while} adalah :

\begin{minted}{cpp}
do {
	pernyataan;
} while(kondisi);
\end{minted}

Diminta masukan PIN melalui \eng{keyboard}. Jika PIN yang dimasukan tidak sama dengan PIN yang ditentukan, maka program akan meminta PIN lagi. Diberikan kesempatan 3 kali sebelum akun dibekukan.

\cppfile{code/masukkan_pin.cpp}

\paragraph{Latihan}
\begin{enumerate}
	\item Buatlah sebuah program yang menerima \(n\) sebagai masukan. Lalu meminta \(n\) angka sebagai masukan, dan menjumlahkan semua masukan tersebut. Contoh:

	Masukan:

	\begin{minted}{cpp}
	4
	5 6 -1 10
	\end{minted}

	Keluaran:

	\begin{minted}{cpp}
	5 + 6 - 1 + 10 = 20
	\end{minted}
\end{enumerate}
\end{document}

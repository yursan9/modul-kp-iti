\documentclass[../main.tex]{subfiles}

\begin{document}
\chapter{Pengaturan Alur Program - Switch}
\section{\emph{Switch}}
Pada praktikum sebelumnya kita belajar tentang mengatur alur program menggunakan
\emph{if} dan rangkaiannya. Sekarang, kita akan belajar tentang \emph{switch}
yang memiliki fungsi yang sama untuk mengatur alur program.

\begin{minted}{cpp}
switch (var){
case ekspresi-konstan 1:
    aksi 1;
    break;
case ekspresi-konstan 2:
    aksi 2:
    break;
default: // Opsional
    pernyataan jika case tidak ada yang tepat;
    break;
}
\end{minted}

Cara kerja dari \emph{switch} adalah membandingkan nilai dalam \code{var} dengan
ekspresi konstan ke-1, ke-2, \ldots{}, ke-\(n\) sampai nilai perbandingan yang
dihasilkan adalah \code{true}. Jika sudah menemukan perbandingannya, maka
pernyataan pada \emph{case} \(n\) tersebut akan dijalankan, dan \emph{case} yang
lain tidak akan diperdulikan.

\cppfile{code/ucap_angka.cpp}

Jika kalian perhatikan dengan seksama, baris-baris di bagian \emph{switch} bisa
kita tulis ulang dengan \emph{if, else if,} dan \emph{else} dan menghasilkan
program yang sama.

\begin{minted}{cpp}
if (angka == 1)
        cout << "Satu!";
else if (angka == 2)
    cout << "Dua!";
else if (angka == 3)
    cout << "Tiga!";
else{
    cout << "Tidak ada pilihan " << angka << endl;
    // Program keluar karena terjadi kesalahan
    return 1;
}
\end{minted}

\paragraph{Latihan}
\begin{enumerate}
  \item Buat sebuah program yang menerima masukkan angka 1--12, dan mengeluarkan
  nama bulan sesuai angka urutan yang dimasukkan. Jika pengguna memasukkan angka
  diluar jangkauan, tampilkan peringatan.
  \item Buat sebuah program yang menentukan tingkat keberhasilan mahasiswa saat
  pengguna memasukkan nilai huruf. Misal, pengguna memasukkan huruf \code{'E'},
  tampilkan bahwa mahasiswa tersebut harus mengulang mata kuliah tersebut.
\end{enumerate}
\end{document}
